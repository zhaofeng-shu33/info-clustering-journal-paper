\documentclass{article}
\begin{document}
\begin{enumerate}
\item What is the main claim of the paper? Why is this an important contribution to the machine learning literature?

    We propose a graph-based info-clustering method based on a multivariate information metric.
   
    The hierarchical tree can be computed efficiently using
    our improved algorithm, which is one order of magnitude faster than previous methods.
    Besides clustering analysis, our method can be adopted and
    used in many other unsupervised data mining tasks including community discovery, link prediction and outlier detection.
    
	There are few existing hierarchical clustering methods with non-binary tree structure. Our proposed hierarchical clustering method can generate non-binary hierarchical tree that can reveal the intrinsic structures in the data, and is robust against outliers. Our method enriches the existing
	hierarchical clustering method galleries.

\item What is the evidence you provide to support your claim? Be precise.

	By theoretical analysis we show that graph-based info-clustering is a special and useful model of info-clustering.
	By both the theoretical analysis and experiments we show that our proposed improvement of principal sequence of partition algorithm is
	much faster than existing ones. By empirical study we show that graph-based info-clustering can be applied to solve outlier detection, community detection,
	link prediction and clustering analysis problems.

\item What papers by other authors make the most closely related contributions, and how is your paper related to them?

    My paper is closed related with \cite{ic2016}, \cite{huang2017information}, \cite{narayanan} and \cite{mac}.
    The theoretical part of my paper is derived from the combination of \cite{ic2016} and \cite{huang2017information}.
    The improved algorithm proposed in this manuscript is based on \cite{narayanan} and \cite{mac}.
    The empirical study borrows some ideas from \cite{mac} to use rbf kernel as weight for clustering analysis.
 
\item Have you published parts of your paper before, for instance in a conference? If so, give details of your previous paper(s) and a precise statement detailing how your paper provides a significant contribution beyond the previous paper(s).

Yes, I have published parts of this submitted manuscript in \cite{zhao2019info}.
This previous conference paper focuses on how to apply graph-based info-clustering to outlier detection problems and propose an improved algorithm for principal
sequence of partition. In this journal manuscript, we have extended the results of graph-based info-clustering for other problems including community
detection, link prediction and clustering analysis. Besides, we have given a solid theoretical analysis for graph-based info-clustering.
\end{enumerate}
\bibliographystyle{plain}
\begin{thebibliography}{9}
	\bibitem{ic2016} Chan C, Al-Bashabsheh A, Zhou Q, Kaced T, Liu T (2016) Info-clustering: A
	mathematical theory for data clustering. IEEE Transactions on Molecular,
	Biological and Multi-Scale Communications 2(1):64--91,
\bibitem{huang2017information}
Huang SL, Makur A, Zheng L, Wornell GW (2017) An information-theoretic approach
to universal feature selection in high-dimensional inference. In: 2017 IEEE
International Symposium on Information Theory (ISIT), IEEE, pp 1336--1340	
	\bibitem{narayanan}
	Narayanan H (1991) The principal lattice of partitions of a submodular
	function. Linear Algebra and its Applications 144:179--216
	\bibitem{mac}
	Nagano K, Kawahara Y, Iwata S (2010) Minimum average cost clustering. In: NIPS
	23, Curran Associates, Inc., pp 1759--1767
	\bibitem{zhao2019info}
	Zhao F, Ma F, Li Y, Huang SL, Zhang L (2019) Info-detection: An
	information-theoretic approach to detect outlier. In: International
	Conference on Neural Information Processing, Springer, pp 489--496
	
\end{thebibliography}
\end{document}