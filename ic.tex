%%%%%%%%%%%%%%%%%%%%%%% file template.tex %%%%%%%%%%%%%%%%%%%%%%%%%
%
% This is a general template file for the LaTeX package SVJour3
% for Springer journals.          Springer Heidelberg 2010/09/16
%
% Copy it to a new file with a new name and use it as the basis
% for your article. Delete % signs as needed.
%
% This template includes a few options for different layouts and
% content for various journals. Please consult a previous issue of
% your journal as needed.
%
%%%%%%%%%%%%%%%%%%%%%%%%%%%%%%%%%%%%%%%%%%%%%%%%%%%%%%%%%%%%%%%%%%%
%
% First comes an example EPS file -- just ignore it and
% proceed on the \documentclass line
% your LaTeX will extract the file if required
\begin{filecontents*}{example.eps}
%!PS-Adobe-3.0 EPSF-3.0
%%BoundingBox: 19 19 221 221
%%CreationDate: Mon Sep 29 1997
%%Creator: programmed by hand (JK)
%%EndComments
gsave
newpath
  20 20 moveto
  20 220 lineto
  220 220 lineto
  220 20 lineto
closepath
2 setlinewidth
gsave
  .4 setgray fill
grestore
stroke
grestore
\end{filecontents*}
%
\RequirePackage{fix-cm}
%
%\documentclass{svjour3}                     % onecolumn (standard format)
%\documentclass[smallcondensed]{svjour3}     % onecolumn (ditto)
\documentclass[smallextended]{svjour3}       % onecolumn (second format)
%\documentclass[twocolumn]{svjour3}          % twocolumn
%
\smartqed  % flush right qed marks, e.g. at end of proof
%
\usepackage{graphicx}
%
% \usepackage{mathptmx}      % use Times fonts if available on your TeX system
%
% insert here the call for the packages your document requires
%\usepackage{latexsym}
% etc.
%
% please place your own definitions here and don't use \def but
% \newcommand{}{}
%
% Insert the name of "your journal" with
\journalname{Machine Learning}
%
\begin{document}

\title{Graph-Based Info-Clustering and its Applications%\thanks{Grants or other notes
%about the article that should go on the front page should be
%placed here. General acknowledgments should be placed at the end of the article.}
}
%\subtitle{Do you have a subtitle?\\ If so, write it here}

%\titlerunning{Short form of title}        % if too long for running head

\author{First Author         \and
        Second Author %etc.
}

%\authorrunning{Short form of author list} % if too long for running head

\institute{F. Author \at
              first address \\
              Tel.: +123-45-678910\\
              Fax: +123-45-678910\\
              \email{fauthor@example.com}           %  \\
%             \emph{Present address:} of F. Author  %  if needed
           \and
           S. Author \at
              second address
}

\date{Received: date / Accepted: date}
% The correct dates will be entered by the editor


\maketitle

\begin{abstract}
We propose a graph-based hierarchical clustering method based on a multivariate information metric.
The proposed method can generate non-binary hierarchical tree that reveals the intrinsic structures in the data, with no hyper-parameter to tune. 
The hierarchical tree can be computed efficiently by 
our improved algorithm.
Experiments show that the clustering result outperforms
other hierarchical clustering techniques and is very suitable to find the complex community structure.
\keywords{First keyword \and Second keyword \and More}
% \PACS{PACS code1 \and PACS code2 \and more}
% \subclass{MSC code1 \and MSC code2 \and more}
\end{abstract}

\section{Introduction}
\label{intro}
% background information goes here
Clustering Analysis is an important research topic in machine learning.
It has wide applications in many areas. For example, to track the epidemic
sources within a community and assist for the model of natural disasters with observed data.

There are many different methods for clustering analysis. One of them is called
hierarchical clustering. The classical method uses the Euclid distance between samples
as the metric to merge the most similar pairs until all the samples are merged into a clustering
tree \cite{slink}. Compared with other clustering method, hierarchical clustering can better describe the relationship between different clusters. Still, the classical hierarchical clustering method has some
drawbacks. One of them is pairwise merging, which is not optimal since it does not consider information shared among more than two clusters. Besides, the criterion to split the clustering tree into a flat clustering structure is arbitrary set and lacks explanations, which bring some mysteries in practical
usage.

In recent 20 years, there are many emerging hierarchical clustering methods which brought new insights into this area. These new methods arise from combination of different domain knowledge and
subject branch. To name but a few, hierarchical density estimation \cite{hde}, Bayesian hierarchical clustering \cite{bhc},
clusterpath based on convex optimization \cite{hocking2011clusterpath},
graph partition based on Dasgupta's cost \cite{dasgupta2016cost},
Gradient-based Hierarchical Clustering based on hyperbolic gemetry \cite{hyperbolic} etc.
These methods have both advantages and disadvantages.
In practice researchers need to choose the proper method according to the specific problem.
For example, an extension called Bayesian Rose Trees of Bayesian clustering
can produce non-binary hierarchical tree
under a given probabilistic model \cite{blundell2011discovering} and have XXX advantages.
But Bayesian model has many hyper-parameters to tune and the clustering results between different runs vary greatly because of its intrinsic randomness.
This method is suitable for problems which XXX.

Besides the hierarchical methods mentioned above, there is a specific branch which brings information theory into clustering. This branch is based on the observation that metrics in hierarchical clusterings
are mostly empirically inspired and may not reflect the true underlining probalistic model of the data.
For such method, the metric of mutual information is used and an agglomerative scheme is adopted \cite{mim}. The mutual information between samples and cluster labels is estimated using Parzen' window, which is a kind of Gaussian mixture model.

\section{Section title}
\label{sec:1}
\subsection{Subsection title}
\label{sec:2}
as required. Don't forget to give each section
and subsection a unique label (see Sect.~\ref{sec:1}).
\paragraph{Paragraph headings} Use paragraph headings as needed.
\begin{equation}
a^2+b^2=c^2
\end{equation}

% For one-column wide figures use
% \begin{figure}
% Use the relevant command to insert your figure file.
% For example, with the graphicx package use
%  \includegraphics{example.eps}
% figure caption is below the figure
% \caption{Please write your figure caption here}
% \label{fig:1}       % Give a unique label
% \end{figure}
%
% For two-column wide figures use
% \begin{figure*}
% Use the relevant command to insert your figure file.
% For example, with the graphicx package use
 % \includegraphics[width=0.75\textwidth]{example.eps}
% figure caption is below the figure
% \caption{Please write your figure caption here}
%\label{fig:2}       % Give a unique label
%\end{figure*}
%
% For tables use
% \begin{table}
% table caption is above the table
% \caption{Please write your table caption here}
% \label{tab:1}       % Give a unique label
% For LaTeX tables use
% \begin{tabular}{lll}
% \hline\noalign{\smallskip}
% first & second & third  \\
% \noalign{\smallskip}\hline\noalign{\smallskip}
% number & number & number \\
% number & number & number \\
% \noalign{\smallskip}\hline
% \end{tabular}
% \end{table}


%\begin{acknowledgements}
%If you'd like to thank anyone, place your comments here
%and remove the percent signs.
%\end{acknowledgements}


% Authors must disclose all relationships or interests that 
% could have direct or potential influence or impart bias on 
% the work: 
%
% \section*{Conflict of interest}
%
% The authors declare that they have no conflict of interest.


% BibTeX users please use one of
\bibliographystyle{spbasic}      % basic style, author-year citations
%\bibliographystyle{spmpsci}      % mathematics and physical sciences
%\bibliographystyle{spphys}       % APS-like style for physics
\bibliography{exportlist.bib}   % name your BibTeX data base

% Non-BibTeX users please use
%\begin{thebibliography}{}
%
% and use \bibitem to create references. Consult the Instructions
% for authors for reference list style.
%
%\bibitem{RefJ}
% Format for Journal Reference
%Author, Article title, Journal, Volume, page numbers (year)
% Format for books
%\bibitem{RefB}
%Author, Book title, page numbers. Publisher, place (year)
% etc
%\end{thebibliography}

\end{document}
% end of file template.tex

